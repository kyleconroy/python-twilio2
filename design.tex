\documentclass{article}
\usepackage{fullpage}
\usepackage{graphicx}
\usepackage{hyperref}

\title{\texttt{twilio} Python Package}
\author{ Kyle Conroy }
\begin{document}
\maketitle

\section{Module Overview}

The \texttt{twilio} package is broken up into three different modules:
\begin{itemize}
\item The \texttt{api} module
\item The \texttt{twiml} module
\item The \texttt{util} module

\section{\texttt{api} Module}

TBD

\section{\texttt{twiml} Module}

The \texttt{twiml} module is responsible for the creation and validation of TwiML. Configuration options also allow users to specift defaults for various verbs.

\subsection{Creation}

TwiML creation begins with the \texttt{Response} verb. Each succesive verb is created by calling various methods on the response, such as \texttt{say} or \texttt{play}. These methods return the verbs they create to ease the creation of nested TwiML. To finish, call the \texttt{toxml} method on the \texttt{Response}, which returns raw TwiML.

\begin{verbatim}
from twilio import twiml

r = twiml.Response()
r.say("Hello")
r.toxml() # returns <Response><Say>Hello</Say><Response>
\end{verbatim}

The verb methods (outlined in the complete reference) take the body (only text) of the verb as the first argument. All attributes are keyword arguements.

\begin{verbatim}
from twilio import twiml

r = twiml.Response()
r.play("monkey.mp3", loop=5)
r.toxml() # returns <Response><Play loop="3">monkey.mp3</Play><Response>
\end{verbatim}

\subsubsection{Psuedo-blocks}

Python 2.6+ added the \texttt{with} statement for context management. Using \texttt{with}, the module can \textit{almost} emulate Ruby blocks.

\begin{verbatim}
from twilio import twiml

r = twiml.Response()
r.say("hello")
with r.gather(end_on_key=4) as g:
    g.say("world")
r.toxml() 

# returns 
# <Response>
#   <Say>Hello</Say>
#   <Gather endOnKey="4"><Say>World</Say></Gather>
# </Response>
\end{verbatim}

\subsection{Configuration}

Usres may want to configure TwiML creation at a global scope. For example, make all \textt{Redirects} use the POST method by default. The \texttt{config} function allows for easy custimization. Each verb also has a config method. See the complete reference for all the configuration options.

\begin{verbatim}
from twilio import twiml

twiml.config(validation=False)
twiml.redirect.config(method=POST)
\end{verbatim}

\subsection{Vaidation}

TBD

\section{Complete Reference}

Below is the complete reference for all the available verbs in the \texttt{twiml} module

\subsection{\texttt{twilio.twiml.Verb} Class}

\subsubsection{\texttt{Verb.say(body, [


\end{document}
